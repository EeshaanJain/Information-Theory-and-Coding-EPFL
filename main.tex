\documentclass[a4paper]{report}
\usepackage{bbold}
\usepackage[a4paper, total={6in, 10in}]{geometry}
\input{./preamble.tex}
% \usepackage{cmbright}
\tikzset{node distance=2.5cm, % Minimum distance between two nodes. Change if necessary.
         every state/.style={ % Sets the properties for each state
           semithick,
           fill=gray!10},
         initial text={},     % No label on start arrow
         double distance=4pt, % Adjust appearance of accept states
         every edge/.style={  % Sets the properties for each transition
         draw,
           ->,>=stealth',     % Makes edges directed with bold arrowheads
           auto,
           semithick}}
\usepackage{faktor}
\usepackage{def}
\usepackage{pgffor}
\makeatletter
\DeclareRobustCommand*{\mfaktor}[3][]
{
   { \mathpalette{\mfaktor@impl@}{{#1}{#2}{#3}} }
}
\newcommand*{\mfaktor@impl@}[2]{\mfaktor@impl#1#2}
\newcommand*{\mfaktor@impl}[4]{
   \settoheight{\faktor@zaehlerhoehe}{\ensuremath{#1#2{#3}}}%
   \settoheight{\faktor@nennerhoehe}{\ensuremath{#1#2{#4}}}%
      \raisebox{-0.5\faktor@zaehlerhoehe}{\ensuremath{#1#2{#3}}}%
      \mkern-4mu\diagdown\mkern-5mu%
      \raisebox{0.5\faktor@nennerhoehe}{\ensuremath{#1#2{#4}}}%
}
\makeatother

\DeclareMathOperator{\Ker}{ker}
\DeclareMathOperator{\im}{Im}

\title{Information Theory and Coding}
\begin{document}
    \maketitle
    \tableofcontents
    % start lessons
    \foreach \lecno in {1,2,...,20}{%
        \edef\FileName{lectures/lecture\lecno}
        \IfFileExists{\FileName}{%
            \par
            \input{\FileName}
        }
        
    }
    
    %\chapter{Lecture 1}
\section{The Flow of Information}
% \begin{figure}
%     \centering


% \end{figure}
% \begin{tikzpicture}

% % draw rectangle node
% 	\node[draw,
% 		minimum width=2cm,
% 		minimum height=1cm,
%             fill=green!50] at (0,0){Information Source};

% % Change line color
% 	\node[draw,
%             fill=orange!50,
% 		minimum width=2cm, 
% 		minimum height=1cm,
%             shape=circle] at (3,0) {Transmuter};

% % Change filling color
% 	\node[draw,
% 		minimum width=2cm,
% 		minimum height=1cm,
%             fill=green!50] at (6,0){Channel};

% % Change text color
% 	\node[draw,
% 		text=red,
% 		minimum width=2cm, 
% 		minimum height=1cm] at (3,-2) {Process 2};

% \end{tikzpicture}

\section{Source Coding}
\begin{definition}
Given an alphabet $\Ucal$ with $|\Ucal| < \infty$. A source code $c$ is a mapping from the alphabet to the set of all finite binary strings represented as $\binset = \{null, 0, ,1, 00, 01, 10, 11, 000, \dots\}$ \ie\ $c:\Ucal \to \binset$.
\end{definition}
\begin{eg}
Consider $\Ucal = \{a,b,c,d,e\}$. Then we can have a $c:\Ucal \to \binset$ as
$c(a) = 0,$ $c(b) = 0,$ $c(c) = null,$ $c(d) = 1,$ $c(e) = 010100$.
\end{eg}
\begin{definition}
A code $c$ is called singular if $\exists u, v \in \Ucal$ s.t. $u\neq v$ but $c(u) = c(v)$.
\end{definition}
\begin{definition}
A code $c$ is called non-singular or injective if it is not singular.
\end{definition}
\begin{eg}
For the alphabet in the previous example, we can have an injective code as $c(a) = 0$, $c(b) = 00$, $c(c) = null$, $c(d) = 1$ and $c(e) = 010100$.
\end{eg}
\begin{definition}
Given $c: \Ucal \to \binset$, we write the concatenation of the codes of two letters $u_1, u_2 \in \Ucal$ as $c(u_1)c(u_2)$. We further define
\[c^2: \Ucal^2 \to \binset \text{ to be } c^2(u_1, u_2) = c(u_1)c(u_2)\]
\[c^n: \Ucal^n \to \binset \text{ to be } c^n(u_1, \dots, u_n) = c(u_1)c(u_2)\dots c(u_n)\]
\[c^*: \Ucal^* \to\binset \text{ to be } c^*(u_1,\dots,u_n) = c(u_1)c(u_2)\dots c(u_n)\]
\end{definition}
\begin{definition}
A code $c:\Ucal\to\binset$ is called uniquely decodable if $c^*$ is injective, \ie\ if $u_1,\dots,u_n \neq v_1\dots,v_k$, then $c(u_1)\dots c(u_n) \neq c(v_1)\dots c(v_k)$.
\end{definition}
\begin{eg}
With the same alphabet as before, we can have $c(a) = 0$, $c(b) = 10$, $c(c) = 110$, $c(d) = 1110$ and $c(e) = 1111$. This code is uniquely decodable and is called a prefix free code.
\end{eg}
\begin{definition}
A code $c:\Ucal\to\binset$ is called prefix free if $\forall u,v \in \Ucal$ s.t $u\neq v$, $c(u)$ is not a prefix of $c(v)$.
\end{definition}
\begin{theorem}
If a code $c$ is prefix free (p.f.), then it is uniquely decodable (u.d.).
\end{theorem}
\begin{proof}
We prove the above theorem by contradiction. Consider there exists a code $c$ which is p.f. but not u.d and we have $u_1, \dots, u_n, v_1, \dots, v_k \in \Ucal$ such that $u_1, \dots, u_n \neq v_1, \dots, v_k$ and $c(u_1)\dots c(u_n) = c(v_1)\dots c(v_k)$. Traversing the code of the sequence of alphabets from the left, we immediately notice that if $\ell(u_1) > \ell(v_1)$ then $c(v_1)$ is a prefix of $c(u_1)$. Similarly if $\ell(u_1) < \ell(v_1)$, $c(u_1)$ is a prefix of $c(v_1)$. But since the code is prefix free, we arrive at a contradiction.
\end{proof}
\begin{definition}
Given a code $c: \Ucal \to \binset$, we define the Kraft Sum
\[\ks(c) = \sum_{u \in \Ucal} 2^{-\ell(u)}\] where $\ell(u) = \text{length}(c(u))$. In case of confusion, we subscript the length by the code, i.e $\ell_c(u) \equiv \ell(u) = \text{length}(c(u))$
\end{definition}
\begin{eg}
Consider the code for the same alphabet as $c(a) = 0$, $c(b) = 1$, $c(c) = null$, $c(d) = 00$ and $c(e) = 010$. Then we have
\[
    \ks(c) = 2^{-1} + 2^{-1} + 2^0 + 2^{-2} + 2^{-2} = 2.375
\]
\end{eg}
\begin{lemma}
Suppose $c:\Ucal \to \binset$ and $d:\Vcal \to \binset$ are two codes, and denote $(c\times d): \Ucal \times \Vcal \to \binset$ as $(c\times d)(u,v) = c(u)d(v)$. Then 
\[\ks(c\times d) = \ks(c)\ks(d)\]
\end{lemma}
\begin{proof}
By definition, we have
\begin{align*}
    \ks(c\times d) &= \sum_{u,v} 2^{-\ell_{c\times d}((u,v))} = \sum_{u,v} 2^{-[\ell_c(u) + \ell_d(v)]} \\
    &= \sum_{u,v} 2^{-\ell_c(u)} \times 2^{-\ell_d(v)} = \sum_u 2^{-\ell_c(u)} \sum_v 2^{-\ell_d(v)} = \ks(c)\ks(d)
\end{align*}
\end{proof}
\begin{corollary}
$\ks(c^n) = [\ks(c)]^n$
\end{corollary}
\begin{theorem}[Kraft's Inequalities]
Suppose $c: \Ucal \to \binset$ is injective, then
\[\ks(c) \leq \log_2(1+|\Ucal|)\]
Suppose $c$ is uniquely decodable, then
\[\ks(c) \leq 1\]
Suppose $c$ is prefix free, then
\[\ks(c) \leq 1\]
\end{theorem}
\begin{proof}
Suppose that $c$ is injective, and let $|\Ucal| = k$ and $\Ucal = \{1, 2, \cdots, k\}$. To maximize the Kraft Sum, we put $c(1) = null$, $c(2) = 0$, $c(3) = 1$ and so on. Also, let $k = 1 + 2 + 4 + 8 + \cdots + 2^{m-1} + r$ where $0 \leq r < 2^m$. Then, we can write
\[\ks(c) = 2 + 2 \times 2^{-1} + 2^2 \times 2^{-2} + \cdots + 2^{m-1}\times 2^{-(m-1)} + r \times 2^{-m} = m + r \cdot 2^{-m}\]
Now, consider $\log_2(1+k) = \log_2(1 + 2^{m} - 1 + r) = \log_2(2^m + r)$. We can then write \[\log_2(1+k) \leq \log_2(2^m(1+r\cdot2^{-m})) = m + \log_2(1+r\cdot 2^{-m}) \]
Notice that $r\cdot 2^{-m} \in [0,1)$ and using the fact that $\log_2(1+x) \geq x$ for $x\in[0,1]$, we get
\[m + \log_2(1+r2^{-m}) \geq m + r\cdot2^{-m} = \ks(c)\]
Having proved for $c$ being injective, we now show it for $c$ being uniquely decodable. First observe that $c$ is u.d. $\iff$ $c^*$ is injective $\implies \forall n,$ $c^n$ is injective. Hence, we can write
\[
[\ks(c)]^n = \ks(c^n) \leq \log_2(1+k^n)
\]
Now, notice that the LHS is exponentially growing in $\ks(c)$ while the RHS is linearly growing in $\ks(c)$. Since exponential growth cannot be dominated by linear growth, we notice that the term growing exponentially must be less than or equal to 1. Thus, we have $\ks(c) \leq 1$.

\noindent
Finally, since every prefix free code is uniquely decodable, the last inequality follows immediately.
\end{proof}
\begin{remark}
Although the Kraft Sum for uniquely decodable codes is less than or equal to 1, the reverse doesn't hold true. Take the code $\Ucal = \{a,b,c\}$ and the code $c$ to be $c(a) = 0$, $c(b) = 00$ and $c(c) = 00$. It is clear that $c$ is not uniquely decodable, but --
\[
\ks(c) = 2^{-1} + 2^{-2} + 2^{-2}  = 1 \leq 1
\]
\end{remark}
    % end lessons
\end{document}